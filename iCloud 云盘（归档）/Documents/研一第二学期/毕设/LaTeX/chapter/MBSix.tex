\chapter{总结与展望}
\section{论文工作总结}
随着互联网的发展以及餐饮行业消费者需求多样性的提高,基于互联网的餐厅点单系统应运而生,很多点单系统只简单实现了点餐功能,无法满足顾客对隐私性、安全性、个性化的要求,彭庆福餐厅点单系统从顾客和商家两个用户角度做出了不同的设计与实现,使用二次认证方式保证顾客隐私安全,推出到店点餐、预约点餐、外卖点餐三种类型供顾客下单,极大程度上带给用户方便快捷的点单体验,并为商家提供了座位管理、统计报表等功能,帮助商家降低管理餐厅的成本和精力,保存海量数据记录。

本文结合了国内外餐厅点单系统的发展状况与研究现状,介绍了彭庆福餐厅点单系统的设计,并使用了以下技术进行实现:
\begin{itemize}
    \item 后端主要使用Spring Boot和Spring Cloud,程序结构为典型的MVC结构。
    \item 使用React与Redux、Immutable.js、antd、React-AMap实现了前端界面的编写。
    \item 使用了Redis记录用户下载缓存、Session缓存。
    \item 使用Druid数据库连接池进行数据关系存储。
    \item 使用Maven工具进行项目管理、编译打包,使用了Gitlab进行项目版本管理及代码存储,从持续集成系统CI获取构建部署包。
\end{itemize}

本文对该系统进行了用例分析和架构设计,阐述了彭庆福餐厅点单系统的六个模块:订单发布模块、支付模块、用户管理模块、统计报表模块、座位管理模块和出品发布模块。在此基础上,借助详细设计类图、时序图、代码介绍和运行截图,对各模块进行了详细设计和实现的具体描述,介绍了系统的线上线下测试环境和功能测试,从而保障系统功能上的安全性和稳定性。

彭庆福餐厅点单系统自从上线使用以来,一直在平稳运行中,用户量逐步增长,并且有多家餐厅并行使用,包括一些连锁店、分店和个体店。截至2020年3月,合作的餐厅已经达到16家,注册用户超过18万人,餐厅实现了存储海量菜品、座位、订单、报表等信息。该系统基本满足了顾客与商家对于该类软件的需求。\\

\section{展望}
目前,本文介绍的彭庆福餐厅点单系统已经可以让顾客进行完整的下单操作,让商家对餐厅进行有效的管理,但系统在功能设计与会员制度上还有着比较大的进步空间。从顾客的角度来说,系统还可以增加菜品收藏、拼单推荐、在线排队、优惠券赠送、虚拟货币充值等功能;从商家角度来说,系统还可以增加会员充值赠送、会员积分、订单集体派送等功能,进一步完善系统的实用性、易用性以及个性化定制。

在后期的需求调研中我们发现顾客对系统的满意度还有待提升,主要因为没有得到很好的使用指南。不少顾客到店后依然选择叫服务员点单,没有发现餐桌上可以自主扫码点单的标示,后期会提升座位二维码以及使用教程的醒目度。商家会通过线下推广宣传以及线上优惠券、团购等服务来吸引更多的顾客关注餐厅,使用该系统。