\chapter{系统测试}
本章基于彭庆福餐厅点单系统的设计与实现给出相应的测试介绍,主要包括测试与开发环境、部分测试的用例分析、测试的设计与实现以及测试结果。从而保障系统功能上的可用性、安全性、稳定性等,方便系统上线、使用和后期功能的扩展。
\section{系统测试环境}
\subsection{测试与开发环境}
本系统的测试环境主要有两部分,包括线下开发环境与预上线环境。线下开发环境部署在本地电脑上,环境配置如表~\ref{table:textEnvironment}所示,后端编码使用JDK1.8环境,IDEA作为集成开发工具,前端编码使用VS Code,使用Maven进行包管理,Gitlab作为代码变更管理,单元测试使用JUnit单元测试框架。

\begin{table}[htbp!]\footnotesize
    \centering
    \caption{线下开发与测试环境}
    \vspace{2mm}
    \begin{tabular}{cp{0.6\columnwidth}}
    \toprule
    \textbf{设备与软件}&\textbf{备注}\\
    \midrule 
    \textbf{系统}& MacOs Mojava 10.14.5\\
    \hline
    \textbf{磁盘}& Macintosh HD 256GB\\
    \hline
    \textbf{处理器}& 2.3GHZIntel Core i5(四核)\\
    \hline
    \textbf{语言}& JDK 1.8.0.191\\
    \hline
    \textbf{开发测试软件}& IntelliJ IDEA 2018.3、VS Code、Chorme Debug、JUnit\\
    \hline
    \textbf{项目管理软件}& Maven 3.5.0、Gitlab\\
    \bottomrule
    \end{tabular}
    \label{table:textEnvironment}
\end{table}

\begin{table}[htbp!]\footnotesize
    \centering
    \caption{预上线集成与测试环境}
    \vspace{2mm}
    \begin{tabular}{cp{0.6\columnwidth}}
    \toprule
    \textbf{设备与软件}&\textbf{备注}\\
    \midrule 
    \textbf{系统}& Linux系统\\
    \hline
    \textbf{磁盘}& 60GB\\
    \hline
    \textbf{处理器}& 四核\\
    \hline
    \textbf{内存}& 16GB\\
    \hline
    \textbf{机型}& Docker虚拟机\\
    \hline
    \textbf{软件部署}& Nginx,TomCat、Jenkins\\
    \bottomrule
    \end{tabular}
    \label{table:textEnvironment2}
\end{table}

线下环境开发完成并进行过系统单元测试之后,基本功能在本地已经验证过,之后便可以部署到预上线环境,进行相关功能模块的测试与验证。
预上线环境配置如表~\ref{table:textEnvironment2}所示,与实际生产环境基本一致,都是使用Docker对部署组件服务和业务逻辑服务进行容器化部署。通过与实际生产一致的环境部署,对系统功能做进一步的测试,尽可能找出并修改系统中潜在的bug及缺陷。\\

\subsection{测试思路}
测试是为了保证系统的各个模块功能流程正常,保证用户在使用过程的流畅性、准确性。系统一共分为两部分环境测试,一种是线下开发测试,一种是预上线集成测试。
\begin{itemize}
    \item 线下开发测试:测试各个功能模块,覆盖到系统所有模块,检验功能实现是否满足业务逻辑的要求,在设计与代码实现上是否存在缺陷或漏洞。测试会对各个功能模块的边界值、类型、返回结果等做检验,保证各接口功能的正常通信,使得系统在使用时功能逻辑正常且运行稳定。
    \item 预上线集成测试:模拟实际生产环境对系统的运行性能、压力进行测试。找出一些线下无法发现,但是影响线上用户体验感的漏洞,可以快速修复性能上的一些bug,对系统作出整体评估。
\end{itemize}

\section{系统测试过程与结果}
\subsection{功能测试}
本节将根据彭庆福点单系统主要的功能性需求部分进行用例测试的描述。

1.到店多次下单功能测试
\begin{table}[htbp!]
    \footnotesize
    \centering
    \caption{到店多次下单功能测试用例}
    \vspace{2mm}
    \begin{tabular}{cp{11.5cm}}
     \hline
     \ 测试用例编号 & TN1 \\ 
     \hline
     \ 测试内容 & 顾客针对同一座位多次下单 \\ 
     \hline
     \ 测试功能 & 系统能够识别用户的多次扫码,保证一个座位同一时刻只能存在单个订单 \\ 
     \hline
     \multirow{4}{*}{测试步骤}
      & 1.	顾客扫桌上二维码,选择菜品下单\\
      & 2.	系统下单成功后,显示订单详情\\
      & 3.	顾客再次扫描该座位二维码\\
      & 4.	系统显示顾客已有订单的详情\\
      & 5.	更换不同顾客扫描该座位二维码\\
      & 6.	系统提示该座位已有订单,无法下单\\
     \hline
     \multirow{1}{*}{预期结果}
      & 1. 顾客连续扫码时,系统能够识别并显示该顾客未完成的订单内容,顾客可以对该订单加菜、结账等\\
      & 2. 当其他顾客在该座位扫码时,系统能够识别出该座位号有未结账的订单,提示顾客更换座位点单\\
    \hline
    \end{tabular}
    \label{table:tn1}
\end{table}

如表~\ref{table:tn1}所示,这一部分的功能测试主要针对顾客多次扫码下单以及不同顾客对同一座位扫码下单的情况,看系统能否有效进行拦截,并给出不同的提示信息。

2.偏远地区外卖点单功能测试

如表~\ref{table:tn2}所示,这一部分的功能测试主要针对订单发布模块,看系统能否有效拦截恶意外卖点单。
\begin{table}[htbp!]
    \footnotesize
    \centering
    \caption{偏远地区外卖点单功能测试用例}
    \vspace{2mm}
    \begin{tabular}{cp{11.5cm}}
     \hline
     \ 测试用例编号 & TN2 \\ 
     \hline
     \ 测试内容 & 顾客选择的配送地点超过商家配送范围 \\ 
     \hline
     \ 测试功能 & 系统能阻止顾客下单超过配送范围的订单,外卖订单一旦被商家接单且未超过配送时间,无法取消 \\ 
     \hline
     \multirow{2}{*}{测试步骤}
      & 1.	顾客选择外卖点单,配送地址填写一个偏远地址,填写其余内容并选菜下单\\
      & 2.	系统获取配送地址,提醒用户该餐厅配送范围超出,无法下单 \\
      & 3.	商家接单后,订单界面的取消订单按钮消失 \\
     \hline
     \multirow{1}{*}{预期结果}
      & 1. 系统能够识别出超出配送范围的外卖订单,并进行有效拦截\\
      & 2. 系统能够识别出订单被商家接单且未超过配送时间,并隐藏取消订单按钮\\
    \hline
    \end{tabular}
    \label{table:tn2}
\end{table}

3.顾客微信、支付宝支付功能测试

如表~\ref{table:tn3}所示,这一部分的功能测试主要针对支付模块,顾客可以通过微信、支付宝、浏览器等多种方式进行扫码点餐,不同的客户端环境中系统提供给用户的付款方式不同,因此需要测试显示给用户的支付方式是否与用户实际选择的一致。
\begin{table}[htbp!]
    \footnotesize
    \centering
    \caption{支付功能测试用例}
    \vspace{2mm}
    \begin{tabular}{cp{11.5cm}}
     \hline
     \ 测试用例编号 & TN3 \\ 
     \hline
     \ 测试内容 & 测试顾客支付功能 \\ 
     \hline
     \ 测试功能 & 顾客选择支付方式与系统实际展示的支付方式是否一致 \\ 
     \hline
     \multirow{2}{*}{测试步骤}
      & 1.	顾客通过支付宝扫码下单,进行支付结账\\
      & 2.	系统展示支付宝付款页面 \\
      & 3.	顾客通过微信扫码下单,进行支付结账\\
      & 4.	系统展示微信付款页面 \\
      & 5.	顾客通过浏览器扫码下单,进行支付结账\\
      & 6.	系统提示用户选择支付方式(微信或者支付宝),根据用户选择情况跳转到相应付款页面 \\
     \hline
     \multirow{1}{*}{预期结果}
      & 1. 当扫码源为微信界面时,系统只提供微信支付方式\\
      & 2. 当扫码源为支付宝界面时,系统只提供支付宝支付方式\\
      & 3. 当扫码源为浏览器时,系统提供微信、支付宝支付方式供用户选择,并根据其选择进行相应付款界面跳转\\
    \hline
    \end{tabular}   
    \label{table:tn3}
\end{table}

4.用户角色功能测试

如表~\ref{table:tn4}所示,这一部分的功能测试主要针对用户管理模块。点餐系统中顾客一共包含两种角色:企业员工角色以及普通顾客角色。餐厅会与某些企业开展合作,如果企业员工到店就餐,可以给予不同程度的折扣优惠。因此需要测试用户切换角色进行下单时,系统是否能够准确判定。
\begin{table}[htbp!]
    \footnotesize
    \centering
    \caption{用户角色功能测试用例}
    \vspace{2mm}
    \begin{tabular}{cp{11.5cm}}
     \hline
     \ 测试用例编号 & TN4 \\ 
     \hline
     \ 测试内容 & 顾客选择不同的角色进行下单 \\ 
     \hline
     \ 测试功能 & 系统能根据不同的用户角色给出不同的订单优惠 \\ 
     \hline
     \multirow{4}{*}{测试步骤}
      & 1.顾客选择某公司员工角色,并点菜下单\\
      & 2.系统识别用户角色,在计算用户支付金额时根据该公司的优惠政策进行打折 \\
      & 3.顾客选择普通顾客角色,并点菜下单 \\
      & 4.系统识别用户角色,在计算用户支付金额时不打折\\
      \hline
     \multirow{1}{*}{预期结果}
      & 1. 系统能够准确识别用户角色,并能够准确匹配该角色对应的折扣优惠政策\\
    \hline
    \end{tabular}   
    \label{table:tn4}
\end{table}

5.统计报表功能测试

如表~\ref{table:tn5}所示,这一部分的功能测试主要针对统计报表模块。商家通过统计报表的功能可以查看本餐厅所有菜品的进销存情况,并根据此报表制定采购计划。商家还可以通过统计报表根据日、周、月、年等维度查看营收情况,并根据报表计算餐厅利润情况。需要测试菜品进销存报表、餐厅营收报表是否和数据库中对应的订单流水数据一致,保证商家查看到准确数据。

\begin{table}[htbp!]
    \footnotesize
    \centering
    \caption{统计报表功能测试用例}
    \vspace{2mm}
    \begin{tabular}{cp{11.5cm}}
     \hline
     \ 测试用例编号 & TN5 \\ 
     \hline
     \ 测试内容 & 测试商家的菜品进销存报表以及餐厅营收报表 \\ 
     \hline
     \ 测试功能 & 报表数据应与数据库数据一致 \\ 
     \hline
     \multirow{4}{*}{测试步骤}
      & 1. 商家查看当日菜品进销存报表\\
      & 2. 系统显示报表信息 \\
      & 3. 商家查看当日营收报表 \\
      & 4. 系统显示报表信息\\
      \hline
     \multirow{2}{*}{预期结果}
      & 1. 系统展示的菜品进销存报表信息与当日所有订单流水对应\\
      & 2. 系统展示的营收报表信息与当日所有订单流水对应\\
    \hline
    \end{tabular}   
    \label{table:tn5}
\end{table}

6.出品发布功能测试

如表~\ref{table:tn6}所示,这一部分的功能测试主要针对出品发布模块,商家可以通过每日出品发布查看各菜品的库存情况,防止出现菜品售罄,而用户可以继续下单的情况。这种情况会给顾客带来不好的用户体验,影响餐厅形象。需要保证顾客在非餐厅营业时间内点餐时,系统提示用户当前时间无法用餐,可以预约点单,防止出现用户下单,而餐厅未营业的情况。因此需要测试餐厅发布的出品计划与顾客查看到的餐厅信息是否一致。
\begin{table}[htbp!]
    \footnotesize
    \centering
    \caption{出品发布功能测试用例}
    \vspace{2mm}
    \begin{tabular}{cp{11.5cm}}
     \hline
     \ 测试用例编号 & TN6 \\ 
     \hline
     \ 测试内容 & 测试商家的出品计划发布功能\\ 
     \hline
     \ 测试功能 & 商家发布的出品计划应与顾客在点单消费过程中查看到的信息一致 \\ 
     \hline
     \multirow{4}{*}{测试步骤}
      & 1. 商家将某一菜品的库存设置为0\\
      & 2. 系统在顾客点单时显示菜品为已售罄\\
      & 3. 商家设置营业时间\\
      & 4. 顾客不在餐厅的营业时间段内点单时,系统需要提醒用户当前时间无法用餐,可以预约点单\\
      \hline
     \multirow{2}{*}{预期结果}
      & 1. 系统展示给顾客的营业时间与商家发布的一致\\
      & 2. 系统展示的菜品库存与商家设置的一致\\
    \hline
    \end{tabular}   
    \label{table:tn6}
\end{table}

7.座位管理功能测试

如表~\ref{table:tn7}所示,这一部分的功能测试主要针对座位管理模块,商家需要将座位信息,包括座位号、对应二维码等信息录入系统,在餐厅座位上张贴对应的二维码,方便顾客点餐。并且商家可以实时查看餐厅座位的占用情况,调整餐厅座位排布,为顾客提供更好的用餐体验。
因此需要测试餐厅的实际占用情况是否和商家查看的座位视图一致。由于座位信息会对应到顾客的订单信息时,还需要测试商家发布的座位信息是否与订单流水中的一致,防止出现二维码与座位不对应的情况。\\

\begin{table}[htbp!]
    \footnotesize
    \centering
    \caption{座位管理功能测试用例}
    \vspace{2mm}
    \begin{tabular}{cp{11.5cm}}
     \hline
     \ 测试用例编号 & TN7 \\ 
     \hline
     \ 测试内容 & 测试餐厅座位管理功能\\ 
     \hline
     \ 测试功能 & 餐厅的实际座位情况与商家查看的座位视图一致 \\ 
     \hline
     \multirow{6}{*}{测试步骤}
      & 1. 商家登记餐厅座位信息,包括座号、二维码、座位数、占用情况等信息\\
      & 2. 系统保存商家修改\\
      & 3. 顾客扫描某一座位上的二维码进行点餐并下单\\
      & 4. 系统下单成功,显示订单信息、座位信息\\
      & 5. 商家查看当前餐厅的座位列表\\
      & 6. 系统显示实时座位列表,展示被占用的座位和空闲座位\\
      \hline
     \multirow{1}{*}{预期结果}
      & 1. 系统展示给商家的座位列表内容与顾客下单情况一致\\
    \hline
    \end{tabular}   
    \label{table:tn7}
\end{table}

\subsection{性能测试}
为了保障顾客在高峰用餐时段的良好用户体验,需要对餐厅点单系统进行性能测试。性能测试分为两个部分,分别是到店点餐性能测试以及外卖点餐性能测试。

由于餐厅座位有限,在进行到店点餐性能测试时,可以参考餐厅最大座位数。目前餐厅座位数为100,对餐厅点单系统模拟每分钟有100个用户请求点餐下单。在进行外卖点餐性能测试时,对餐厅点单系统模拟每分钟有200个用户请求点餐下单。分别测试系统的平均响应时间和内存占用率,衡量系统性能。

在测试过程中,通过Linux的常用监控命令查看CPU占用率,并通过测试工具平台查看接口的平均响应时间。到店点餐的平均响应时间为300ms,CPU占用率小于30\%,外卖点餐的平均响应时间为250ms,CPU占用率小于40\%。测试结果表明系统的性能良好,能够保证在高峰就餐段系统的平稳运行。

\section{本章小结}
本章主要基于第三章中的需求分析对彭庆福餐厅点单系统的测试情况进行介绍。餐厅点单系统分别进行了功能测试和性能测试,测试结果表明该系统已达到需求分析部分提出的目标,并且系统的性能能够满足用户使用要求。