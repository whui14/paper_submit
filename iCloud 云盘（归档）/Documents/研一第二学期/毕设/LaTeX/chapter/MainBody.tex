\chapter{引言}

\section{项目背景}
在中国文化中,饮食文化扮演着重要角色,餐饮业是一个历史悠久的行业。随着社会的不断发展、科技的不断进步,人民的经济水平提高带动了消费水平的发展,大家对吃的形式要求越来越多样,对餐饮服务的需求也从单独的食物质量扩展到服务的便利性、高效性。随着互联网的不断发展,越来越多的人习惯于手机支付而非传统的现金支付,期望实现自助点单、自助结算的一体化服务。餐厅用餐看似是个简单的点单过程,实际上很可能是决定餐厅运营成功与否的关键,因为它不光要求信息传递的快速性与准确性,还需要面向客户服务。随着餐厅运营规模的不断扩充,顾客流量的不断增长,年轻的消费群体变得越来越多,更现代、快捷、时尚的消费观在慢慢形成,普通的基于服务员点餐的运转模式越来越不能够满足当代人需求。

随着餐厅业务的发展与客流量的增多,服务员在面对客户点单时遇到越来越多无法解决的问题,比如人流量太多无法及时为每一位顾客点单,客户需要提前预约下单或者点本店外卖等。
本文所实现的餐厅点餐系统转变了操作对象,从原来的餐厅服务员转移到了消费的顾客群体,这种改变增加了餐厅与顾客之间的互动性,为顾客提供了许多便捷式服务,比如菜品的详情介绍、自助下单与买单、对菜品进行评价等。

虽然有很多点餐平台提供类似服务,但通常无法系统整合餐厅的所有定制化需求,例如会员不同折扣、菜品原材料来源、实体座位与线上座号绑定等。另外,用户数据既关系到个人隐私,又是企业重要的战略资源,公司不希望用户数据外流。因此彭庆福公司急需独立开发一个可以集到店点单、外卖点单、预约点单于一身的餐厅点单平台,使得餐厅可以精准定位每一个用户信息,用户也可以在该餐厅系统中通过充值、成为会员等方式来获得折扣、赠品等优惠,提高服务质量和效率,提升买卖双方的满意度。
系统可以帮助商家管理菜品、订单,帮助顾客节省采购和等待时间,弥补了传统人力管理的多种缺陷,节省了工作人员手工记录的时间和精力,进行海量数据管理与存储。
使用网络操作线上订餐,不仅可以保证订单出品的快速性、准确性,还可以进行数据统计、分析客户私人喜好进行设置。无论对于商家还是顾客,餐厅点单系统的使用都为其带来诸多便利。

\section{在线餐饮系统国内外发展概况}
餐饮行业在国内所采用的管理系统相比国外较为落后,形成原因是多方面的。
一方面相对国外而言,我国的餐饮行业中规模较大的餐饮机构比较少,而规模较小的餐饮机构,大多不愿投入成本去开发及使用管理系统,其相关管理人员也缺乏采用系统进行餐厅管理的意识。
另一方面,由于餐饮行业对员工的教育素养要求较低,即使购买了管理系统,员工学习、使用起来也相对困难,耗费人力的同时工作效率也不会得到太明显的提升。
最后,相对于发达国家,我国互联网技术兴起时间比较晚,在早期开发一个相关的管理系统所耗费的成本比较高,也限制了在餐饮行业进行使用、推广管理系统的发展。
大多关于餐饮行业设计的信息管理系统参照的都是酒楼、餐馆的管理模式,设计的功能大同小异,没有结合具体餐饮店的实际需求进行改进与优化~\cite{DBLP:series/txcs/Voorhees20}。
系统的功能性不够完善、可用性较低,不能显著地提高管理餐厅的工作效率,节省人力资源的消耗。

国外餐饮服务行业的信息管理系统比较先进,这主要受益于其计算机技术开始研发的时间比较早,使得技术水平在当前的研究上处于较高层次,它对信息管理系统的应用十分常见,其设计内容、覆盖范围都相对广泛。
在1970年左右,餐饮行业的管理工作就逐渐使用与网络相关的信息管理系统,国外餐饮行业的发展时间较长,使得其发展相对成熟、稳定,尤其是像肯德基、赛百味、麦当劳等大型连锁餐饮企业比较常见~\cite{yy}。
由于规模较大、分店遍布全球,为了方便管理,这些机构率先使用了餐饮管理信息系统对各个餐饮分支线进行统一的组织与管理。
它们使用餐饮信息管理系统的初衷大多是针对餐厅采购完商品进入仓库,再从仓库中将其分配到各个餐饮分店中,并且进行总店库存中商品数量等的统计,当然这些需求及应用相对比较简单。
随着社会经济的不断发展与互联网技术的不断迭代更新,电子商务这一概念越来越被重视,餐饮行业关于管理上的需求也不再仅仅局限于上述内容,出现了互联网采购商品、对历史记录的搜索查看、统计商品采购报表、管理成本、分析数据等多种需求。

国内现有餐厅点单系统有很多,比如“美味不用等”,它是一个支持线上排队和预约、点餐、支付、管理会员等功能的平台,将重点放在到店消费方向,使得顾客可以线上提前排队,并建立会员制度抓住潜在客户和目标客户,帮助餐厅提升业绩。“美味不用等”与相关大平台基本连接,但是和大部分餐厅管理商尚未实现互通,无法获得预期流量\cite{gcf}。美团、饿了么等外卖订餐平台将线上选购与线下实体店铺相融合,通过提供商户信息、优惠信息、在线预约操作等,使商家与顾客紧密联系。它们在不断发展的同时,也存在着很多不足,比如无法保证线上出售的商品卫生安全达到标准、线上订购的商品与线下送达的商品质量不一、售后服务无法保证顾客权益、相关市场逐渐饱和限制平台发展等~\cite{htxO2O}。

随着经济和互联网的不断进步与网络化企业信息管理的不断发展,餐厅点单系统需要向专业化、多元化发展,保障商品的安全与品质、保障顾客隐私与权益、满足特定客户的特色需求,所以一个符合大部分餐厅需求、有一些定制化功能的餐厅点单管理系统会受到大部分餐饮企业的青睐,并且餐饮企业希望该系统具有上手方便快捷、功能齐全、拥有良好的可扩展性、易于维护等特性~\cite{DBLP:journals/jbi/VelupillaiSLRSM18}。
由于系统良好的维护性和可扩展性,开发者得以在较低成本投入的前提下,在系统原有的功能基础上,为每一个餐饮客户开发出符合大众需要又兼具各自特色的餐厅点单系统。

目前国内外的餐厅在点单方面大多通过服务员帮忙记录手写菜单点餐的传统方式,这种方式错误率较高,不方便存档,明显与当今快节奏的社会生活脱节,采用互联网技术进行管理与应用,是我们向条理化、规范化、统一化餐厅管理系统的一种迈进。
管理系统的引入不仅可以减轻员工的工作负担与工作强度,以此提高他们的工作效率,而且可以提升顾客的满意度,帮助他们提前点单、节省用餐时间,从而带来客观的经济收入。
充分借鉴和利用互联网技术和国外优势,提升顾客就餐感受,努力和国际接轨,是我国餐饮业今后的发展方向与潮流。

% \section{一些正文中的标记}
% \emph{斜体}

% \textbf{加粗}

% \texttt{代码元素格式}

% \begin{center}
% 居中,左右对齐同理。
% \end{center}

% 这里展示脚注。\footnote{数字列举和圆点列举见摘要部分}

% 一个小建议,中文后直接跟上述格式标记(包含各种引用)可能会出现一些问题。
% 因此,在中文字和格式标记的斜杠之间加入~\emph{一个波浪号}是一个常用的习惯。
% 双~~波~~浪~~线等价于一个强制空格,有时比键盘输入的空格要好用。


\section{本文主要工作}
本文主要内容是对彭庆福餐厅点单系统的设计、架构与实现过程的研究。
其主要研究思路是通过使用软件工程方法,去设计和实现整个系统,第一阶段进行功能调研和需求分析,与做需求、产品、运营的同学进行讨论,定位系统的功能与具体权限细节等。
第二阶段进行竞品调研与分析,研究餐饮企业内相关软件的使用与优缺点,进行总结、整合与分析,借鉴其优势并且避免其劣端。
第三阶段,进行功能点的细分与整体构设计,找出合适的技术方案,并画出用例图,给出用例描述。
第四阶段,进行系统开发,编写前后端代码以及测试,并进行部署上线使用。 

彭庆福餐厅点单系统主要面向两个用户:顾客和商家。
每到用餐时间会有大量的顾客通过该系统下单,根据需要选择到店或者外卖点单。商家可以通过该系统获取用户的订单,并根据要求完成订单内容的制作。 

系统主要有三个部分,包括H5端(即支付宝、微信公众号、手机浏览器页面)、Web网页端、后端。预计H5端为主要的流量入口,实现用户多方面点单的需求:普通用户可以进入个人中心查看信息,可以到店点餐、外卖点餐或者预约点餐,可以对已完成的订单菜品进行评价,查看订单列表等;服务员可以管理订单、管理座位、收银、查看统计报表等。
为了统一化界面UI,该系统使用了Ant Design作为产品的设计体系,Immutable管理数据、防止state对象被错误赋值;使用Redis实现了用户下单记录、分布式Session的数据缓存;使用MySQL实现了餐厅信息、收益、结算账单等关系型数据的存储。
该餐厅点单系统需要解决的一大难题就是如何快速地将某一个供餐实体的顾客和商家对应起来,同时能够实现所有订单信息的互通,这主要通过数据库和ORM(Object Relational Mapping)框架丰富的功能来完成。
彭庆福餐厅点单系统对顾客提供了可视化的订餐界面。顾客可以选择用餐方式,并选购菜品生成订单。不同用餐方式的订单将会被录入到统一的数据库中,更加方便进行账单统计。顾客再次进入订餐界面时,系统能够根据鉴权信息,快速获取此前的点单信息,方便加菜或者退菜,所有对订单相关的操作都会被记录下来。 

在商家端,系统往往保持常开状态,商家能够获取此前未处理的所有订单信息。商家需要时刻关注正在处理的订单状态,监听新订单的状态,因此每个商家端管理平台都需要轮询获取自己的订单信息。这可能会产生比较大的流量,需要控制轮询的间隔,从而与实时的订单状态达到一个平衡,在数据库层面需要做好缓存设计。对于处理中的订单,可以通过key来访问数据,使用Session缓存和懒加载机制减少对数据库的访问。对于在生产环境中运行的网络流量,系统设计了监控平台来监控当前的HTTP平均访问延迟和HTTP请求成功率。一旦发现系统的承载过大,会快速降低轮询的间隔,减少系统流量。 

餐厅管理系统对于餐饮行业至关重要,可以提升很多性能,比如降低成本、减少员工数量、提高翻台率、增加毛利润……,
餐厅点单系统的核心价值是提高餐饮企业的管理效率,当顾客有需要可以手机上呼叫,并得到及时回应。
智能手机是一种快捷、全面且方便的工具,有不少商家在使用手机点单系统,其功能越来越全面、个性化定制越来越可以满足客户需求,可以预见手机点单系统势必会成为餐饮软件的一种新趋势。
\section{本文组织结构}
本文的组织结构如下:

第一章是引言部分。从业务、前景、技术等多角度介绍了彭庆福餐厅点单系统的项目背景,并且阐述了其在国内外研究、发展的现状,简要叙述了本文的主要工作。

第二章是技术综述。介绍了彭庆福餐厅点单系统实现过程中所涉技术以及框架,其中包括了Spring MVC、Spring Cloud微服务部署架构、Spring Boot、Druid、Redis、前端技术React和Redux、antd、Webpack、React-AMap、Kafka、微信支付宝支付等技术与相关理论的介绍与原理。 

第三章是彭庆福餐厅点单系统的需求分析和概要设计。借助用例图、需求列表详细分析了系统需求,并介绍了系统的非功能性需求。结合逻辑视图、开发视图、物理视图、流程图、系统架构图、ER图等详细描述了系统总体的架构设计、功能执行流程设计与数据库设计,并介绍了系统的搭建和部署实现。 

第四章是彭庆福餐厅点单系统的详细设计与实现。介绍了系统主要的六个模块,包括发布模块、支付模块、用户管理模块、统计报表模块、座位管理模块和出品发布模块。通过模块结构图分析了它们之间相互依赖的关系,并且通过类图、时序图等展示并详细说明了这些模块的设计与具体实现细节。

第五章是系统测试内容,通过对系统功能测试和性能测试的描述,介绍了系统在开发过程中对功能点的测试情况。 

第六章是总结与展望。总结论文所做的工作与系统上线使用情况,并对彭庆福餐厅点单系统在未来将要优化的方向做了进一步的展望与拓展。

% \section{注意软换行的使用}
% 论文一般会引用代码,本模板建议将代码声明为~\texttt{class.this()}格式。
% 在引用代码时,较长的函数名会导致函数名超出文本边界的情况,因此可以考虑手动进行软换行,请参考以下例子。

% “图XX 展示了从AquaLush 系统中抽取的函数调用依赖示例,其中~\texttt{UICon-} \linebreak \texttt{troller.buildLogScrn()} 是为了实现新功能“the control panel shows log message”而在新版本中添加的函数。”

